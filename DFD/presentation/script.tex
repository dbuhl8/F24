\documentclass{article}

\usepackage{graphicx} % Required for inserting images
\usepackage[left=1in,right=1in,top=1in,bottom=1in]{geometry}
\usepackage{amsmath}
\usepackage{amsthm} %proof environment
\usepackage{amssymb}
\usepackage{amsfonts}
\usepackage{enumitem} %nice lists
\usepackage{verbatim} %useful for something 
\usepackage{xcolor}
\usepackage{setspace}
\usepackage{blindtext} % I have no idea what this is 
\usepackage{caption}  % need this for unnumbered captions/figures
\usepackage{natbib}
\usepackage{tikz}
\usepackage{soul} % need this for the hl command
\usepackage{hyperref}

\begin{document}

\title{The Effects of Rotation on Stratified Turbulence}
\author{Dante Buhl \\{\small Advised by: Pascale Garaud}}
\date{November 25, 2024}


\newcommand{\wrms}{w_{\text{rms}}}
\newcommand{\bs}[1]{\boldsymbol{#1}}
\newcommand{\tb}[1]{\textbf{#1}}
\newcommand{\bmp}[1]{\begin{minipage}{#1\textwidth}}
\newcommand{\emp}{\end{minipage}}
\newcommand{\R}{\mathbb{R}}
\newcommand{\C}{\mathbb{C}}
\newcommand{\N}{\mathcal{N}}
\newcommand{\K}{\bs{\mathrm{K}}}
\newcommand{\m}{\bs{\mu}_*}
\newcommand{\s}{\bs{\Sigma}_*}
\newcommand{\dt}{\Delta t}
\newcommand{\dx}{\Delta x}
\newcommand{\tr}[1]{\text{Tr}(#1)}
\newcommand{\Tr}[1]{\text{Tr}(#1)}
\newcommand{\Div}{\nabla \cdot}
\renewcommand{\div}{\nabla \cdot}
\newcommand{\Curl}{\nabla \times}
\newcommand{\Grad}{\nabla}
\newcommand{\grad}{\nabla}
\newcommand{\grads}{\nabla_s}
\newcommand{\gradf}{\nabla_f}
\newcommand{\xs}{\bs{x}_s}
\newcommand{\xf}{\bs{x}_f}
\newcommand{\ts}{t_s}
\newcommand{\tf}{t_f}
\newcommand{\pt}{\partial t}
\newcommand{\pz}{\partial z}
\newcommand{\uvec}{\bs{u}}
\newcommand{\F}{\bs{F}}
\newcommand{\T}{\tilde{T}}
\newcommand{\ez}{\bs{e}_z}
\newcommand{\ex}{\bs{e}_x}
\newcommand{\ey}{\bs{e}_y}
\newcommand{\eo}{\bs{e}_{\bs{\Omega}}}
\newcommand{\ppt}[1]{\frac{\partial #1}{\partial t}}
\newcommand{\ppts}[1]{\frac{\partial #1}{\partial t_s}}
\newcommand{\pptf}[1]{\frac{\partial #1}{\partial t_f}}
\newcommand{\ppz}[1]{\frac{\partial #1}{\partial z}}
\newcommand{\ddz}[1]{\frac{d #1}{d z}}
\newcommand{\ppzetas}[1]{\frac{\partial^2 #1}{\partial \zeta^2}}
\newcommand{\ppzs}[1]{\frac{\partial #1}{\partial z_s}}
\newcommand{\ppzf}[1]{\frac{\partial #1}{\partial z_f}}
\newcommand{\ppx}[1]{\frac{\partial #1}{\partial x}}
\newcommand{\ppy}[1]{\frac{\partial #1}{\partial y}}
\newcommand{\ppzeta}[1]{\frac{\partial #1}{\partial \zeta}}


\maketitle 

\section{Intro Slide}

\section{Motivation 1}
\begin{itemize}
    \item We are studying rotating stratified turbulence, as stratified
    turbulence plays a crucial role in mixing and vertical transport in many
    geophysical and astrophysical fliud dynamics. 
    \item In many geophysical flows, both stratification and rotation influence
    dynamics. Some popular examples are the oceans' thermocline, and the Solar
    Tachocline. 
\end{itemize}

\section{Motivation 2}
\begin{itemize}
    \item This problem is particularly interesting because of the competing
    nature of stratification and rotation.
    \item Stably stratified flows are typically characterized by strong
    anisotropy, where pancake structures are formed with an aspect ratio
    controlled by the Froude number. 
    \item Rotating flows, by contrast, often form tall cylindrical vorticies
    along the axis of rotation. 
    \item Using DNS, we attempt to study these competing dynamics and their
    affect on mixing in the flow. 
\end{itemize}

\section{Schematic}
\begin{itemize}
    \item Our DNS will use a tripply periodic domain with horizontal lengths of
    $4\pi$ in x and y, and a vertical length of $\pi$. 
    \item This domain will have a linear background Temperature profile which is
    hot at the top and cool at the bottom providing stable stratification. 
    \item The rotation axis will be aligned with the z-axis and gravity which
    effectively means we are studying the effect of rotation at the poles. 
\end{itemize}

\section{Governing Equations}
\begin{itemize}
    \item Our governing equations are the standard incompressible Boussinesque
    equations which have been nondimensionalized
    using a characteristic velocity $U$, a large-eddy horizontal length scale
    $L$, a buoyancy frequency $N$, planetary angular velocity $\Omega$, and
    viscous and thermal diffusivity coefficients $\nu$ and $\kappa$
    respectively. 
    \item We are able to define the Reynolds, Peclet, Froude, and Rossby numbers
    according to this nondimensionalization. And in the DNS that follow, we have
    fixed the Reynolds number to 600 and Peclet number to $60$, which implies a
    Prandtl number of $0.1$. 
\end{itemize}

\section{Forcing Mechanism}
\begin{itemize}
    \item Finally, we employ a stochastic forcing mechanism which is purely
    horizontal and divergence free. 
    \item This forcing will be employeed in spectral space and will therefore be
    dependent on the wavenumber. Specifically, we chose only to force horizontal
    wavenumbers which are in absolute value less than or equal to $\sqrt{2}$,
    which implies a minimum forcing lengthscale of a little less than half of
    the domain.
    \item The stochastic process used is a Gaussian process prescribed to be of
    amplitude 1 and correlation timescale 1. 
    \item (Optional) similar forcing mechanisms have been used by several
    studies of stratified turbulence in the past c.f. Waite and Bartello
    (2004)/(2006)
\end{itemize}

\section{Non-rotating Stratified Turbulence}
\begin{itemize}
    \item Before studing the effect of rotation, we conducted DNS in order to
    validate this forcing mechanism as we have employeed it. 
    \item The images on this plot show the x-component of the velocity field
    along the top of the domain and the front of the domain in the top and
    bottom rows respectively. Note that the strength of the stratification
    increases from left to right. 
    \item Consistent with prior studies of stratified turbulence, we see that
    the flow becomes increasingly anisotropic as the Froude number decreases. 
    \item Now that we have confirmed that this forcing mechanism produces
    stratified turbulence we are ready to study the effect of rotation. 
\end{itemize}

\section{Rotating Stratified Turbulence at Fixed $Fr = 0.18$}
\begin{itemize}
    \item Here you see the vertical component of the vorticity field
    along the top of the domain for simulations with varying rotation rate. Note
    that the larger the inverse Rossby number, the more rapidly rotating the
    flow is. 
    \item Notice that in the more weakly rotating simulations (top row), there do not
    appear to be any stable structures which retain their form. For the more
    rapidly rotating simulations (bottom row), a stable vortex has appeared
    corresponding to the areas of strong vertical vortcity shown in red. 
\end{itemize}

\section{Vertically-Invariant structures in the flow ($Fr = 0.18$)}
\begin{itemize}
    \item Further investigation of these vortices reveals that they are indeed
    vertically invariant, that is the vortex penetrates the entire vertical extent
    of the domain. 
    \item Here you see volume renderings of the vertical vorticity shaded
    according to a gaussian transfer function visible in the colorbar on each plot. 
    \item Notice that in the weakly rotating simulation (left) there does not
    appear to be any vortex in the flow, in the moderately rotating simulation
    (middle),
    the vortex appears to be vertically invariant, but the regions outside of
    the vortex are seemingly still turbulent and have non-uniform vertical
    structures. Finally, in the rapidly rotating simulation (right), the region outside
    of the main cyclonic vortex seems to have formed into an anti-cyclone which
    exhibits weak vertical variance. 
    \item What this seems to indicate is that as the rotation rate increases,
    more energy is being put into large-scale horizontal modes of the velocity. 
\end{itemize}

\section{Inverse energy Cascade ($Fr = 0.18$)}
\begin{itemize}
    \item This suspicion is confirmed by the energy spectra for the horizontal
    wavenumbers of the flow. 
    \item Each plot has the absolute horizontal wavenumber along the x-axis and
    the energy along the y-axis. Furthemore, the horizontal energy is shown in red, and the vertical
    energy is shown in blue. Each plot has a red line which corresponds to a
    $|\bs{k}_h|^{-5/3}$ energy spectrum, and a blue dashed line which
    corresponds to the smallest horizontal wavenumber which is forced in each
    simulation. 
    \item Inspecting these spectra, we deduce that the simulation with inverse
    Rossby of 1, has an energy spectrum very similar to the non-rotating case
    and this confirms that the effect of rotation on this simulation is minimal. 
    \item For the two spectra presented from higher inverse Rossby number, we
    see an increase of energy in the smallest horizontal wavenumbers, indicative
    of an inverse energy cascade. Furthermore, we see that for the simulation
    with the highest inverse rossby number shown the horizontal energy spectrum
    seems to follow the $|\bs{k}_h|^{-5/3}$ very nicely (with the exception of
    the forcing also providing some additional energy). 
\end{itemize}

\section{R.M.S. Data}
\begin{itemize}
    \item transition sentence
    \item These plots depict time-averaged rms quantities of the total
    horizontal velocity (left) and the vertical velocity (right) from different
    simulations as they vary with inverse Rossby number. The series of blue
    points correspond to the simulations with $Fr = 0.18$ and the points in red
    correspond to the simulations with $Fr = 0.1$. Finally, the full circles
    represent data taken from simulations which are statistically stationary,
    and the hollow circles represent data taken from simulations which had not
    yet reached a statistically stationary state. 
    \item As expected due to the inverse energy cascade, the horizontal rms
    velocity seems to increase proportionally to the inverse rossby number and
    doesn't seem to depend on the Froude number.
    \item The vertical velocity which for small inverse Rossby number, is
    dependent on the Froude number, seems to become less dependent on the
    Froude number as the inverse Rossby number inreases. The data suggests that
    for moderate rotation rates the vertical rms velocity is roughly constant
    and may decrease for strong rotation rate. 
    \item This is a bit odd at first, since as the rotation rate increases we
    expect there to be less vertical energy in the flow. 
\end{itemize}

\section{Vertically-Averaged Flow}
\begin{itemize}
    \item In order to investigate this further, we used a vertical average to
    understand how the vortices affect the flow. 
    \item Here we show the vertically averaged planetary vorticity (top row) and
    vertical velocity squared (bottom row) from simulations of varying rotation
    rates. 
    \item For the weakly rotating simulation (left) there doesn't appear to be
    any correlation between the planetary vorticity and squared vertical
    velocity. For the moderately rotating simulation (middle), in the vortex,
    where the planetary vorticity is strongest, there seems to be a void in the
    squared vertical velocity. And in the stronger rotating simulation (right),
    vertical motion seems to be entirely restricted to the anti-cyclone which
    coresponds to the region of near zero planetary vorticity. 
    \item This gives us a much better understanding of where vertical motions
    can take place in the flow. 
\end{itemize}

\section{Temperature Transport and Mixing in the Flow}
\begin{itemize}
    \item Next we see if the Temperature transport and mixing are affected by
    the vortex in the same way. 
    \item Similar to the plot of the vertical rms velocity shown earlier, both
    the Temperature Flux (left), and mixing efficiency (right) seem to remain constant for
    weak and moderate rotation rates. This is likely due to the fact that the
    volume fraction of the domain which inhibit vertical motions is still
    relatively small.  
    \item I should note that the mixing efficiency for these simulations,
    is rather high, and thats simply because we are in the Low Prandtl Number
    limit (i.e. the flow is very thermally diffusive)
\end{itemize}

\section{Correspondance between Planetary Vorticity and Mixing}
\begin{itemize}
    \item To confirm the suspicion that mixing is inhibitted by the vortex core
    and vertically-invariant structures within the flow, we compare the
    vertically averaged planetary vorticity to the thermal dissipation within
    the flow. 
    \item Similar to the plots of the squared vertical velocity, we see that
    thermal dissipation is inhibitted within the vortex core for the
    moderately rotating simulation (middle) and is strictly limited to the
    anti-cyclone for the strongly rotating simulation (right). 
\end{itemize}

\section{Summary}
\begin{itemize}
    \item To conclude, what we have learned from this work is the following:
    \item For $Ro \le 1$, there is no significant change from the non-rotating
    DNS. 
    \item For $1 > Ro > Fr$, the flow becomes increasingly two dimensional and
    vertical mixing is localized to regions of low planetary vorticity. 
    \item For particularly low Rossby numbers, the cyclones are especially
    stable, and mixing is exclusively restricted to the anti-cyclones
    within the flow. 
    \item For $Ro > Fr$, the mixing efficiency is approximately constant. 
\end{itemize}

\end{document}


