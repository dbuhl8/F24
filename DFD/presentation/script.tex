\documentclass{article}

\usepackage{graphicx} % Required for inserting images
\usepackage[left=1in,right=1in,top=1in,bottom=1in]{geometry}
\usepackage{amsmath}
\usepackage{amsthm} %proof environment
\usepackage{amssymb}
\usepackage{amsfonts}
\usepackage{enumitem} %nice lists
\usepackage{verbatim} %useful for something 
\usepackage{xcolor}
\usepackage{setspace}
\usepackage{blindtext} % I have no idea what this is 
\usepackage{caption}  % need this for unnumbered captions/figures
\usepackage{natbib}
\usepackage{tikz}
\usepackage{soul} % need this for the hl command
\usepackage{hyperref}

\begin{document}

\title{The Effects of Rotation on Stratified Turbulence}
\author{\small \textbf{Dante Buhl} Pascale Garaud, Hongyun Wang}
\date{November 25, 2024}


\newcommand{\wrms}{w_{\text{rms}}}
\newcommand{\bs}[1]{\boldsymbol{#1}}
\newcommand{\tb}[1]{\textbf{#1}}
\newcommand{\bmp}[1]{\begin{minipage}{#1\textwidth}}
\newcommand{\emp}{\end{minipage}}
\newcommand{\R}{\mathbb{R}}
\newcommand{\C}{\mathbb{C}}
\newcommand{\N}{\mathcal{N}}
\newcommand{\K}{\bs{\mathrm{K}}}
\newcommand{\m}{\bs{\mu}_*}
\newcommand{\s}{\bs{\Sigma}_*}
\newcommand{\dt}{\Delta t}
\newcommand{\dx}{\Delta x}
\newcommand{\tr}[1]{\text{Tr}(#1)}
\newcommand{\Tr}[1]{\text{Tr}(#1)}
\newcommand{\Div}{\nabla \cdot}
\renewcommand{\div}{\nabla \cdot}
\newcommand{\Curl}{\nabla \times}
\newcommand{\Grad}{\nabla}
\newcommand{\grad}{\nabla}
\newcommand{\grads}{\nabla_s}
\newcommand{\gradf}{\nabla_f}
\newcommand{\xs}{\bs{x}_s}
\newcommand{\xf}{\bs{x}_f}
\newcommand{\ts}{t_s}
\newcommand{\tf}{t_f}
\newcommand{\pt}{\partial t}
\newcommand{\pz}{\partial z}
\newcommand{\uvec}{\bs{u}}
\newcommand{\F}{\bs{F}}
\newcommand{\T}{\tilde{T}}
\newcommand{\ez}{\bs{e}_z}
\newcommand{\ex}{\bs{e}_x}
\newcommand{\ey}{\bs{e}_y}
\newcommand{\eo}{\bs{e}_{\bs{\Omega}}}
\newcommand{\ppt}[1]{\frac{\partial #1}{\partial t}}
\newcommand{\ppts}[1]{\frac{\partial #1}{\partial t_s}}
\newcommand{\pptf}[1]{\frac{\partial #1}{\partial t_f}}
\newcommand{\ppz}[1]{\frac{\partial #1}{\partial z}}
\newcommand{\ddz}[1]{\frac{d #1}{d z}}
\newcommand{\ppzetas}[1]{\frac{\partial^2 #1}{\partial \zeta^2}}
\newcommand{\ppzs}[1]{\frac{\partial #1}{\partial z_s}}
\newcommand{\ppzf}[1]{\frac{\partial #1}{\partial z_f}}
\newcommand{\ppx}[1]{\frac{\partial #1}{\partial x}}
\newcommand{\ppy}[1]{\frac{\partial #1}{\partial y}}
\newcommand{\ppzeta}[1]{\frac{\partial #1}{\partial \zeta}}


\maketitle 

\section{Intro Slide}
\begin{itemize}
    \item Mention advisor/collaborator
\end{itemize}

\section{Motivation 1}
\begin{itemize}
    \item This talk is motivated by the phenomenon of stratified turbulence,
    which plays a crucial role in mixing and vertical transport in many
    geophysical and astrophysical flows. 
    \item In many such flows, both stratification and rotation influence
    dynamics. Some popular examples are the oceans' thermocline, and the Solar
    Tachocline. 
\end{itemize}

\section{Motivation 2}
\begin{itemize}
    \item This problem is particularly interesting because of the competing
    nature of stratification and rotation.
    \item Stably stratified flows are typically characterized by strong
    anisotropy, where pancake structures are formed with an aspect ratio
    controlled by the Froude number. 
    \item Rotating flows, by contrast, often form tall cylindrical vorticies
    along the axis of rotation. 
    \item Using DNS, we attempt to study these competing dynamics and their
    affect on mixing in the flow. 
\end{itemize}

\section{Schematic}
\begin{itemize}
    \item Our DNS will use a tripply periodic domain with horizontal lengths of
    $4\pi$ in x and y, and a vertical length of $\pi$. 
    \item This domain will have a linear background Temperature profile which is
    hot at the top and cool at the bottom providing stable stratification. 
    \item The rotation axis will be aligned with the z-axis and gravity which
    effectively means we are studying the effect of rotation at the poles. 
\end{itemize}

\section{Governing Equations}
\begin{itemize}
    \item Our governing equations are the standard incompressible Boussinesque
    equations which have been nondimensionalized
    using a characteristic velocity $U$, a large-eddy horizontal length scale
    $L$, a buoyancy frequency $N$, planetary angular velocity $\Omega$, and
    viscous and thermal diffusivity coefficients $\nu$ and $\kappa$
    respectively. 
    \item We are able to define the Reynolds, Peclet, Froude, and Rossby numbers
    according to this nondimensionalization. And in the DNS that follow, we have
    fixed the Reynolds number to 600 and Peclet number to $60$, which implies a
    Prandtl number of $0.1$. 
\end{itemize}

\section{Forcing Mechanism}
\begin{itemize}
    \item Finally, we employ a stochastic forcing mechanism which is purely
    horizontal and divergence free. 
    \item This forcing will be employeed in spectral space and will therefore be
    dependent on the wavenumber. Specifically, we chose only to force horizontal
    wavenumbers which are in absolute value less than or equal to $\sqrt{2}$,
    which implies a minimum forcing lengthscale of a little less than half of
    the domain.
    \item The stochastic process used is a Gaussian process prescribed to be of
    amplitude 1 and correlation timescale 1. 
    \item Similar forcing mechanisms have been used by prior
    studies of stratified turbulence in the past c.f. Waite and Bartello
    (2004)/(2006)
\end{itemize}

\section{Non-rotating Stratified Turbulence}
\begin{itemize}
    \item Before studing the effect of rotation, we conducted DNS in order to
    validate this forcing mechanism as we have employeed it. 
    \item The images on this plot show a horizontal component of the velocity field
    along the top of the domain (top row) and the front of the domain (bottom
    row). Note that the strength of the stratification increases from left to right. 
    \item Consistent with prior studies of stratified turbulence, we see that
    the aspect ratio of the flow decreases with the Froude number. 
    \item Now that we have confirmed that this forcing mechanism produces
    stratified turbulence we are ready to study the effect of rotation. 
\end{itemize}

\section{Rotating Stratified Turbulence at Fixed $Fr = 0.18$}
\begin{itemize}
    \item Here you see the vertical component of the vorticity field
    along the top of the domain for simulations with varying rotation rate. Note
    that the larger the inverse Rossby number, the more rapidly rotating the
    flow is. 
    \item Notice that in the more weakly rotating simulations, there do not
    appear to be any stable structures. In the more
    rapidly rotating simulations, a stable vortex with a large horizontal
    lengthscale has formed and seems to become domain filling. 
\end{itemize}

\section{Vertically-Invariant structures in the flow ($Fr = 0.18$)}
\begin{itemize}

    \item Volume renderings of the vertical vorticity reveal that these are
    vertically invariant cyclones which take up the entire vertical extent of
    the domain. 
    \item Notice that in the slowly rotating simulation (left) there does not
    appear to be any cyclone in the flow, in the moderately rotating simulation
    (middle),
    the cyclone appears to be vertically invariant, but the regions outside are still
    turbulent and have non-uniform vertical structure. 
    Finally, in the rapidly rotating simulation (right), the region outside
    of the main cyclonic vortex seems to have formed into an anti-cyclone which
    exhibits decreased vertical variance. 
    \item This reveals that more energy is being put into large-scale horizontal
        modes as the rotation rate increases. 
\end{itemize}

\section{Inverse energy Cascade ($Fr = 0.18$)}
\begin{itemize}
    \item This suspicion is confirmed by plots of the energy spectra of the flow. 
    \item These are plots of the horizontal and vertical energies against the total
        horizontal wavenumber showon in red and blue respectively. In each plot
        there is a red $|\bs{k}_h|^{-3}$ spectrum overplotted and a dashed blue line
        signifying the smallest forced wavenu,ber. 
    \item There are two essential takeaways from these plots, first that the
        slowly rotating flow seems to be negligably affected, and that the
        rapidly rotating flows experience an inverse energy cascade, gathering
        energy at the smallest horizontal wavenumbers. 
\end{itemize}

\section{R.M.S. Data}
\begin{itemize}
    \item Looking at some of the more quantitative data, we also compare the
    total horizontal and vertical rms velocities for different Froude and Rossby
    numbers. The solid circles on these plots represent time-averaged data from
    a statistically-stationary state, while the open circles are taken from
    non-stationary states. Higher inverse Rossby number data is not included on
    this plot as we do not believe it is close to a stationary state yet. 
    \item The total horizontal velocity seems to increase proportionally to the 
    inverse rossby number starting at $Ro = 1$ for both Froude numbers. 
    \item The vertical velocity remains roughly constant, $\to$ interesting
    because its dependence on the Froude number becomes less
    dominant as the inverse Rossby number increases, and the appearance of
    cyclones suggest there should be less vertical movement in the flow.
\end{itemize}

\section{Vertically-Averaged Flow}
\begin{itemize}
    \item To investigate this further, we used a vertical average to
    understand how the cyclones affect mixing and vertical transport in the flow. 
    \item Here we show the vertically averaged total vorticity (top row) and
    squared vertical velocity (bottom row) from simulations of varying rotation
    rates. 
    \item The weakly rotating simulation (left) doesn't appear to have any interesting
    correlation between vorticity and vert velocity which is what we expect as
    there is no stable cyclone in the flow. The moderately rotating simulation
    (middle), demonstrates a void in the vertical velocity precisely where the
    cyclone is concentrated. Finally, the rapidly rotating simulation appears to
    have little to no vertical motions except where the total vorticity is near
    zero. This corresponds to an anti-cyclone which forms in the rapidly
    rotating simulations. 
\end{itemize}

\section{Thermal Dissipation and Mixing in the Flow}
\begin{itemize}
    \item The next item to investigate if thermal dissipation and mixing are
    affected in the same way. 
    Similar to vertical velocity, the rms Temperature Flux (left), and mixing
    efficiency (right) seem to remain constant for weak and moderate rotation rates. 
    \item I should note that the mixing efficiency for these simulations,
    is rather high, and thats simply because we are in the Low Prandtl Number
    limit (i.e. the flow is very thermally diffusive)
\end{itemize}

\section{Correspondance between Total Vorticity and Mixing}
\begin{itemize}
    \item Similar to the plots of the squared vertical velocity, we see that
    thermal dissipation is inhibitted within the vortex core for the
    moderately rotating simulation (middle) and is strictly limited to the
    anti-cyclone for the strongly rotating simulation (right). 
\end{itemize}

\section{Summary}
\begin{itemize}
    \item To conclude, what we have learned from this work is the following:
    \item For $Ro \le 1$, there is no significant change from the non-rotating
    DNS. 
    \item For $1 > Ro > Fr$, the flow becomes increasingly two dimensional and
    vertical mixing is localized to regions of low planetary vorticity. 
    \item For particularly low Rossby numbers, the cyclones are especially
    stable, and mixing is exclusively restricted to the anti-cyclones
    within the flow. 
    \item For $Ro > Fr$, the mixing efficiency is approximately constant. 
\end{itemize}

\end{document}


